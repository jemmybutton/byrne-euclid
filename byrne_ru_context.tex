\input preamble.tex
\input preamble_be.tex

\def\inpropstr{пр.}
\def\inpoststr{пост.}
\def\indefstr{опр.}
\def\inaxstr{акс.}
\def\qedstr{Ч. Т. Д.}

\starttext
\startbook[title={Книга I}]
\startVerboseProposition[title={Предложение I. Задача}, reference=prop:I.I]

\defineNewPicture[1/2]{
	pair A, B, C;
	path P[];
	numeric r;
	r := 3/2u;
	A := (0, 0);
	B := (r, 0);
	P1 := fullcircle scaled 2r;
	P2 := fullcircle scaled 2r shifted B;
	C := P1 intersectionpoint P2;
		byLineDefine(A, B, black, 0, 0);
		byLineDefine(B, C, byred, 0, 0);
		byLineDefine(C, A, byyellow, 0, 0);
		draw byNamedLineSeq(1)(AB,CA,BC);
		draw byCircle(A, B, byblue, 0, 0, 1/2)(A);
		draw byCircle(B, A, byred, 0, 0, 1/2)(B);
		draw byLabelsOnPolygon(A, C, B)(0, -1);
}
\drawCurrentPictureInMargin
\problemNP{Н}{а}{данной ограниченной прямой \drawUnitLine{AB} построить равносторонний треугольник.}

\startCenterAlign
Опишем \offsetPicture{15pt}{0pt}{\drawFromCurrentPicture{
draw byNamedLine(AB);
draw byNamedCircle(A);
draw byLabelLineEnd(A, B, 0);
draw byLabelLineEnd(B, A, 1);
}} и \offsetPicture{15pt}{0pt}{\drawFromCurrentPicture{
draw byNamedLine(AB); 
draw byNamedCircle(B);
draw byLabelLineEnd(A, B, 1);
draw byLabelLineEnd(B, A, 0);
}}
\inpost[post:III];\\
проведем \drawUnitLine{CA} и \drawUnitLine{BC} \inpost[post:I].\\
Тогда \drawLine[bottom][triangleABC]{AB,CA,BC} равносторонний.

Поскольку $\drawUnitLine{AB} = \drawUnitLine{CA}$ \indef[def:XV];\\
и $\drawUnitLine{AB} = \drawUnitLine{BC}$ \indef[def:XV];\\
$\therefore \drawUnitLine{CA} = \drawUnitLine{BC}$ \inax[post:I];\\
и значит \triangleABC\ и есть искомый треугольник.
\stopCenterAlign

\qed
\stopVerboseProposition

\startProposition[title={Предложение II. Задача}, reference=prop:I.II]
\defineNewPicture{
pair A, B, C, D, E, F;
path P[];
numeric r[];
A := (0, 0);
B := (-3/5u, -3/5u);
C := (-2u, -1/3u);
r1 := abs(A-B);
D := (fullcircle scaled 2r1 shifted A) intersectionpoint (fullcircle scaled 2r1 shifted B);
r2 := abs(B-C);
r3 := r1 + r2;
P1 := fullcircle scaled 2r2 shifted B;
P2 := fullcircle scaled 2r3 shifted D;
E := (D -- 10[D, B]) intersectionpoint P1;
F := (D -- 10[D, A]) intersectionpoint P2;
byLineDefine(A, B, black, 1, 0);
byLineDefine(B, C, black, 0, 0);
byLineDefine(B, D, byred, 0, 0);
byLineDefine(D, A, byred, 0, 0);
byLineDefine(B, E, byyellow, 0, 0);
byLineDefine(A, F, byblue, 0, 0);
draw byNamedLineSeq(0)(AF,DA,BD,BE);
draw byNamedLineSeq(0)(AB,BC);
draw byCircle(D, E, byred, 0, 0, 1/2)(A);
draw byCircle(B, C, byblue, 0, 0, -1/2)(B);
draw byLabelsOnPolygon(E, D, A, F)(2, -1);
draw byLabelsOnPolygon(E, B, C)(2, -1);
draw byLabelLineEnd(C, B, 0);
draw byLabelLineEnd(E, B, 1);
draw byLabelLineEnd(F, A, 1);
}
\drawCurrentPictureInMargin
\problemNP{О}{т}{данной точки \drawUnitLine{DA,AF} отложить прямую равную данной прямой \drawUnitLine{BC}.}

\startCenterAlign
Проведем \drawUnitLine{AB} \inpost[post:I], построим \drawFromCurrentPicture[bottom]{
startAutoLabeling;
startTempScale(scaleFactor*3);
startGlobalRotation(180-lineAngle.AB);
draw byNamedLineSeq(0)(AB,BD,DA);
stopGlobalRotation;
stopTempScale;
stopAutoLabeling;
} \inprop[prop:I.I],\\
продлим \drawUnitLine{BD} \inpost[post:II],\\
опишем
\drawFromCurrentPicture{
draw byNamedLine (BC); 
draw byNamedCircle(B); 
draw byLabelLineEnd(B, C, 0); 
draw byLabelLineEnd(C, B, 0);
}
\inpost[post:III], и
\drawFromCurrentPicture{
draw byNamedLine (BD, BE);
draw byNamedCircle(A);
draw byLabelLineEnd(D, E, 0); 
draw byLabelLineEnd(E, D, 1);
}
\inpost[post:III];\\
продлим \drawUnitLine{DA} \inpost[post:II],\\
тогда искомая прямая это \drawUnitLine{AF}.

Поскольку $\drawUnitLine{BE,BD} = \drawUnitLine{DA,AF}$ \indef[def:XV],\\
и $\drawUnitLine{BD} = \drawUnitLine{DA}$ (постр.),\\
$\therefore \drawUnitLine{BE} = \drawUnitLine{AF}$ \inax[post:III],\\
но \indef[def:XV] $\drawUnitLine{BC} = \drawUnitLine{BE} = \drawUnitLine{AF}$;

$\therefore \drawUnitLine{AF}$ проведенная из данной точки (\drawUnitLine{DA,AF}) равна данной прямой \drawUnitLine{BC}.
\stopCenterAlign

\qed
\stopProposition

\stopbook
\stoptext
%\closeout \lettrineslist
